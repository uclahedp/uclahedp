\documentclass[12pt]{article}

\usepackage[letterpaper,margin=1in]{geometry}

\usepackage{hyperref}
\usepackage{tcolorbox}

\usepackage{graphics}
\graphicspath{ {images/} }

\newcommand{\loc}[1]{
{\bf \fontfamily{pcr}\selectfont #1}
}

\newcommand{\todo}[1]{ \begin{tcolorbox} \centering  #1 \end{tcolorbox}}

\newcommand{\key}[3]{{\loc{#1}} (#2) : #3}


\title{UCLA HEDP Experimental Plasma Physics Analysis Package Documentation}
\date{\today}
\author{Peter Heuer}


\begin{document}
\emergencystretch 3em

\maketitle

\newpage

\tableofcontents

\newpage

\section{Overview}

\begin{figure}[h]
\centering
\includegraphics[width=\textwidth]{process_flow}
\caption[Processing Workflow]
{\label{process_flow} A visualization of the processing pipeline.}
\end{figure}

The UCLA HEDP package provides a pipeline for turning  raw datafiles from a number of sources into fully-processed data ready for analysis. The package also includes a toolkit of analysis routines intended to reduce code duplication for common tasks like filtering data. Throughout the pipeline all data is stored in a common format (the CDF format) which enables all data at any point to be plotted using a common plotting routine, provided by the \loc{dataview} program.

The processing pipeline takes datafiles from several sources as inputs, as well as metadata CSV files that adhere to a standard format. In the first stage of processing, these various data sources are merged with metadata and saved in a standardized format. No physics is accounted for in this stage. In the second stage, these "raw" files are transformed into "full" files. In this stage, raw voltage values are transformed into physically meaningful outputs ready for higher level analysis. The abstraction of the "raw" processing step from the "full" step allows data from a Langmuir probe (for example) hooked up to any data source to be processed by the same Langmuir probe processing routines.


\subsection{Concepts}

Throughout the pipeline (and the associated metadata CSV format), data is labeled using two important concepts:

\begin{itemize}
\item \textbf{probes}: A probe is a a string that labels a particular measurement source. A probe could have an associated position (like a bdot) or not (like a timing photodiode). A probe can correspond to any number of digitizer channels. A probe can have metadata associated with it either permanently or only for a single experimental run. 

\item \textbf{runs}: A run is a repetition of the experiment (or multiple repetitions) that belong conceptually together and are therefore stored and processed together. A run can involve multiple probes. A run can have metadata associated with it (independent of any probes). A run is represented by a run number.
\end{itemize}

\begin{figure}[h]
\centering
\includegraphics[width=\textwidth]{file_struct}
\caption[File Structure]
{\label{file_struct} A recommended file structure for storing data.}
\end{figure}

When storing datafiles for an experiment, it is recommended (but not strictly required) for the following system of sub-directories to be used:

\begin{itemize}

\item \textbf{METADATA}: This directory is where the package will search for any metadata CSV files. All files in this directory (and in sub-directories recursively) will be searched. 

\item \textbf{HDF}: This is where the original HDF files are stored.

\item \textbf{RAW}: This is where output of the raw or "load" processing step is stored.

\item \textbf{FULL}: This is where the the fully processed datafiles are output.

\end{itemize}

All of the routines within the package take a full filepath to specify the data to load in and write out, so the data need not be stored together (except for metadata, which must be collected in a directory so it can be searched). 



\subsection{Philosophy}

A few guiding principles that guided the design choices behind the structure of the UCLA HEDP package include:

\begin{itemize}

\item \textbf{Only original datafiles and metadata should need to be stored}.\\ Processed datafiles should be easy enough to create that they can be re-created when needed from the original datafiles and metadata. This principle saves disk space and also ensures that the data being used was processed using up-to-date metadata (as opposed to a saved processed datafile whose origins may be unclear). 

\item \textbf{When possible, data should be stored in a common format that can be easily plotted.}\\ Storing data in a common format (in this case the CDF format) enables standardized plotting routines which make it easy to check data at different stages of analysis. 

\item \textbf{Minimize the amount of data in RAM at any time}. Any computer should be able to process any dataset, given sufficient time. Programs should take advantage of the HDF file format to only load parts of the data in memory during processing so that even laptop computers can process 100 GB datafiles.

\item \textbf{Reduce code duplication through standardization}. A bdot is the same probe whether it is used in LAPD or in the Phoenix lab, and it should therefore be processed using the same bdot routines.

\end{itemize}


\section{The Common Data Format (CDF)\label{cdf}}

\begin{figure}[h]
\centering
\includegraphics[width=0.7\textwidth]{hdf5_struct}
\caption[Example of CDF HDF5 Files]
{\label{hdf5_struct} Example structures of a raw and a full HDF5 file that conform to the CDF.}
\end{figure}


\subsection{Requirements}

The UCLA HEDP package is designed around a standardized structured data format, hereafter referred to as common data format or CDF. A CDF object is an HDF file that conforms to a standard structure as defined in the function \loc{tools.dataset.validDataset}. In particular, each CDF file must contain

\begin{itemize}

\item A CDF object is a group in an HDF file. It may be the root group, or multiple CDF objects may be housed within a HDF file as separate groups.

\item A dataset called "data" with $n$ dimensions. This array must have an attribute "dimensions" which holds a string array of list $n$ which provides a name for each dimension in the dataset (by convention, all lowercase). A second attribute "unit" defines the units of the data in this array as a string that can by interpreted by the \loc{astropy.units} string-to-unit parser. 

\todo{In the future this could be generalized so that the CDF object could contain multiple data arrays with different names (but all with the same shape and the same axes). This would be conducive to datasets such as density and electron temperature planes from Langmuir measurements which share a set of axes and belong conceptually together. In this case, a new attribute array should be added to the parent group listing the names of all of the data arrays. }

\item For each entry in the "dimensions" attribute there must exist a dataset with the same name which contains the axis values for that dimension. This dataset must also include a mandatory "unit" attribute as defined above.

\item Other attributes (commonly including metadata) are included as attributes of the root group in the HDF file. These entries should be tuples of the form (value, unit) where unit is a unit string as described in the previous points.

\end{itemize}

This format was designed to contain all of the information necessary to make plots of the dataset.

\todo{
One notable exception to the CDF format is probe position information which is saved in an additional array called "pos" with dimensions [shot num, 3] and has no attributes. Other similar non-standard arrays include t0ind and badshots in timing diode files. These "special" files that contain these additional required array should be considered subclasses of the CDF format - they inherit all of the normal requirements but also require these additional arrays.
}

\subsection{Conventions}
\begin{itemize}
\item The convention [shots, time, channels] is commonly used for raw shot-ordered datasets.
\item The convention [time, xaxis, yaxis, zaxis, repetition channel] is commonly used for processed volumetric datasets. 
\end{itemize}

These conventions are chosen to allow easy plotting of subsets of an array. For example, if a dataset has dimensions B = [time, xaxis, yaxis, zaxis, repetition channel]. then it is easy to extract a time trace for a particular position, repetition, and channel B[t] = B[:, x, y, z, r, c] (where x,y,z,r,c represent fixed values for the other axes) or an XY plane B[x,y] = [t, :, :, z, r, c]. 

\subsection{Considerations for Large Datasets}

Large datasets are not stored on consecutive sections of computer memory but are rather broken into "chunks" that are stored separately. Accessing data spread across multiple chunks is much more computationally costly than accessing the same amount of data from a single chunk. The HDF format and the \loc{h5py} package allow control over how a dataset is broken into chunks in order to maximize efficiency. Where possible, the UCLA HEDP package attempts to chunk datasets in a way that minimizes read times for common operations. For example, a dataset containing time arrays at a variety of spatial locations may chose to chunk the time axis together, anticipating that time traces may be accessed more often than large spatial volumes.


\section{The CSV Metadata Format\label{csv_format}}

\subsection{CSV File Header Format}

Metadata is recorded in CSV files in a standard format, and many column header names may be required by different routines in the analysis pipeline:

Across all metadata files the following header format is followed for the top three rows
\begin{enumerate}
\item Machine-readable keywords, eg. "probe", that will become dictionary keys.
\item Unit string, or blank for dimensionless units, eg. "mm2"
\item Human-readable title or note, eg. "Bdot Area". This row is not used by the program but improves readability of the CSV file.
\end{enumerate}

\subsection{The Four Types of Metadata Files}
Metadata files are sorted into four types automatically based on whether the csv file includes the "run" or "probe" keywords (Table~\ref{csv_types}). This system allows information that does not change throughout the experiment for either one or all probes to not be repeated unnecessarily, which keeps the metadata files managable and reduces errors.

\begin{table}[]
\begin{tabular}{p{1cm}p{2cm}p{2cm}p{6cm}}
Type                                 & Has "run" key? & Has "probe" key? & Explanation                                                                          \\ \hline
\multicolumn{1}{c|}{Experiment Type} & No             & No               & Data that pertains to the whole experiment, eg. experiment name, vacuum chamber used \\
\multicolumn{1}{c|}{Run Type}        & Yes            & No               & Data that pertains to a given run number, eg. background field, fill pressure        \\
\multicolumn{1}{c|}{Probe Type}      & No             & Yes              & Data about a particular probe for the whole experiment, eg. calibration constants    \\
\multicolumn{1}{c|}{Run-Probe Type}  & Yes            & Yes              & Data about a probe on a particular run, eg. positon or attenuation                  
\end{tabular}
\caption{Types of CSV spreadsheets as defined by the presence of the "run" and/or "probe" keywords.\label{csv_types}}
\end{table}

A raw file for a particular probe and run will have the following metadata attached from these files:

\begin{itemize}
\item All metadata from all experiment type metadata files.

\item All metadata from all run type metadata files that match the run number.

\item All metadata from all probe type metadata files that match the probe name.

\item All metadata from all run-probe type metadata files that match BOTH the run and the probe.
\end{itemize}


\todo{Tip: Non-integer run numbers can be used to label multiple sub-runs with metadata from the main run. For example, if multiple movies are captured using a camera during run 32, which is a long probe plane, the camera runs could be labeled 32.1, 32.2 etc. and will automatically inherit the run-type metadata from run 32.}



\section{Structure of the Package}
\input{structure}

\appendix 

\section{Metadata CSV Key Dictionary}
This appendix defines a number of dictionary keys.
\subsection{Experiment Data}

No standard keys exist for this type of file yet

\subsection{Run-Type} 

\begin{itemize}
\item \key{datafile}{str}{Datafile name. "datafile".hdf5 should be the datafile for each "datafile" in this column.}
\end{itemize}

\subsection{Probe-Type}

\begin{itemize}
\item \key{probe\_type}{str}{Specifies the probe type, used to decide which analysis routines to run. Example: "bdot", "tdiode".}
\end{itemize}

\subsubsection{Probe-Type: Bdots Probes}

\begin{itemize}
\item \key{area}{float}{Area of the probe tip (used when calculating isat density)}

\end{itemize}

\subsubsection{Probe-Type: Langmuir Probes}

\begin{itemize}
\item \key{nturns}{int}{Number of bdot turns (assume to be the same for all axes).}
\item \key{\{xyz\}area}{float}{Area of the probe tip from calibration.}
\item \key{\{xyz\}tau}{float}{High-frequency calibration constant for each axis.}
\end{itemize}

\subsection{Run-Probe Type}

\begin{itemize}
\item \key{\{xyz\}pol}{1 or -1}{This factor is multiplied by the data that is read in to potentially reverse it if a probe was in upside down.}
\item \key{gain}{float}{Amplifier gain prior to the digitizer.}
\end{itemize}

\subsubsection{Run-Probe Type: LAPD Digitizer}

\begin{itemize}
\item \key{digitizer}{str}{Name of the digitizer used. Example "SIS crate".}
\item \key{adc}{str}{Name of the analog-to-digital converter used. Example "SIS 3305", "SIS 3302".}
\item \key{brd\{i\}}{int}{Digitizer board used, where \{i\} is the number of the channel. There should be one of these columns for each channel.}
\item \key{chan\{i\}}{int}{Channel on the digitizer used, where \{i\} is the number of the channel. There should be a corresponding "brd\{i\}" for each one.}
\end{itemize}

\subsubsection{Run-Probe Type: HRR Digitizer}
\begin{itemize}
\item \key{resource\{i\}}{int}{Resource number for each channel \{i\}.}
\item \key{chan\{i\}}{int}{Channel number. The number of these columns should match the number of resource columns.}
\end{itemize}

\subsubsection{Run-Probe Type: Probes with Position Information}

\begin{itemize}
\item \key{probe\_origin\_\{xyz\}}{float}{Position of the probe origin relative to the experiment coordinate system. These will be added to all the probe positions.}

\item \key{\{xyz\}pos}{float}{Position of the probe. Overridden by motor drive information if a probe is being scanned. These positions are relative to the probe origin.}

\item \key{roll}{float}{Angle, in degrees, that the probe was rotate about its central (x) axis. This is included during the rotation correction phase, and can be used to correct a probe that was misaligned.}

\item \key{rot\_center\_\{xyz\}}{float}{Required only for probes that rotate on a ball valve (like the LAPD probe drives). This specifies the center position of the ball valve, for use in angle corrections.}

\item \key{ax\_pol\_\{xyz\}}{1 or -1}{Direction of the motion axis relative to the experiment coordinate system. Set to -1 if they are anti-parallel. If this keyword is not included, a value of 1 is assumed by default. Currently motion axes at an angle to the experiment coordinate system are not supported (but theoretically could be added...)}

\end{itemize}

\subsubsection{Run-Probe Type: Probes Using LAPD Motor Drives}

\begin{itemize}
\item \key{motion\_controller}{str}{Which LAPD drive was associated with the probe. Example "6K Compumotor", "NI\_XYZ".}

\item \key{motion\_receptacle}{int}{Which instance of that motor drive was used? Starts at 1. For example, 6K Compumotor can control four XY drives, labeled 1,2,3,4".}

\end{itemize}


\subsubsection{Run-Probe Type: Probes Using HRR-Controlled Motor Drives}

\begin{itemize}
\item \key{\{xyz\}pos\_resource}{int}{Resource number for each channel of the probe drive.}
\item \key{\{xyz\}pos\_chan}{int}{Channel number for each channel of the probe drive.}
\end{itemize}

\subsubsection{Run-Probe Type: Bdot Probes}

\begin{itemize}
\item \key{\{xyz\}atten}{float}{Attenuation on each axis of the probe. Required to be in dB currently}
\end{itemize}

\subsubsection{Run-Probe Type: Langmuir Probes}

\begin{itemize}
\item \key{atten}{float}{Attenuation on the digitizer. Required to be in dB currently}

\item \key{ramp\_atten}{float}{Attenuation for ramp, only required for vsweep.}

\item \key{ramp\_gain}{float}{Gain on ramp signal: only required for vsweep. Regular "gain" keyword is used for the other signal channel.}

\item \key{sweep\_type}{str}{Currently "langmuir\_isat" or "langmuir\_vsweep" are used. This keyword is just useful for deciding which type of analysis routine to call on this dataset.}

\item \key{resistor}{float}{Measurement resistor for vsweep runs.}

\item \key{bias}{float}{Probe bias for isat runs.}

\end{itemize}


\end{document}